\documentclass[a4paper]{article}
\usepackage{graphicx}

\usepackage{geometry}
 \geometry{
 a4paper,
 total={210mm,297mm},
 left=35mm,
 right=35mm,
 top=30mm,
 bottom=30mm,
 }
 
\begin{document}

\title{Design, Construction and Performance Tests of a Prototype Micromegas Chamber with Two Readout Layers in a Common Gas Volume}

\maketitle

In recent years, micropattern gaseous detectors received significant attention in the development of precision and cost-effective tracking detectors in nuclear and high energy physics experiments. The important task for these detectors is not only a precise position measurement, but also the determination of the incoming angle of traversing particles. One possible realization, using a single Micromegas readout layer, is the so-called Micro-TPC method. However, its angle resolution is very limited, in particular for perpendicular incident beams.
\\
\\
We therefore designed a new prototype detector based on Micromegas technology, which consists of two readout layers that are placed face to face and share a common gas volume, separated by a common cathode mesh. A schematic illustration is given in Figure 1. The compact structural design has a reduced material budget, compared to two individual Micromegas detectors, and also reduces the number of required support infrastructure, such as gas- or high-voltage connections. In addition, its height of a few mm allows an application in limited space environments. This reduces significantly the multiple scattering of incident particles and hence makes this design also suited for nuclear physics experiments.
\\
The prototype detector is designed to achieve an angle resolution of 0.5$^\circ$ with a spatial resolution of 100 $\mu m$. The performance of the prototype detector is measured at the cosmic ray measurement facility at the University of Mainz. In particular, the reconstruction efficiency, the spatial- and the angle-resolutions are determined for several operation conditions. All performance results will be present in this contribution.
\\
\\
In summary, the developed prototype detector is optimized for a low material budget design and allows for the precise reconstruction of the incident angle of ionizing particles. This opens up a new field for the application of Micromegas detectors in particle physics experiments, especially in high rate environments, where the position and angle resolution is essential to separate signal from background events.

\begin{figure}
    \centering
    \includegraphics[width=5.0in]{ViennaDetector.pdf}
    \caption{Schematic sketch of the double layer MicroMegas detector.}
    \label{simulationfigure}
\end{figure}

\end{document}

